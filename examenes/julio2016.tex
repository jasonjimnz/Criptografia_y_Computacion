\documentclass[10pt,spanish]{article}

\usepackage[spanish]{babel}
\usepackage[utf8]{inputenc}
\usepackage{amsmath, amsthm}
\usepackage{amsfonts, amssymb, latexsym}
\usepackage{enumerate}
\usepackage[usenames, dvipsnames]{color}
\usepackage{colortbl}
\usepackage[left=3cm, right=3cm]{geometry}
\usepackage{cancel}
\usepackage{graphicx}
\usepackage{subfigure}

\usepackage[bookmarks=true,
            bookmarksnumbered=false, % true means bookmarks in
                                     % left window are numbered
            bookmarksopen=false,     % true means only level 1
                                     % are displayed.
            colorlinks=true,
            linkcolor=webblue]{hyperref}
\definecolor{webgreen}{rgb}{0, 0.5, 0} % less intense green
\definecolor{webblue}{rgb}{0, 0, 0.5}  % less intense blue
\definecolor{webred}{rgb}{0.5, 0, 0}   % less intense red

\setlength{\parindent}{0pt}
\setlength{\parskip}{1ex plus 0.5ex minus 0.2ex}

%%%%% Para cambiar el tipo de letra en el título de la sección %%%%%%%%%%%
\usepackage{sectsty}
\sectionfont{\fontfamily{pag}\selectfont}
\subsectionfont{\fontfamily{pag}\selectfont}
\subsubsectionfont{\fontfamily{pag}\selectfont}

\definecolor{morado}{rgb}{0.6, 0.4, 0.8}

% \usepackage[default]{frcursive}
% \usepackage[T1]{fontenc}

%Definimos autor y título
\title{\fontfamily{pag}\selectfont \bfseries \Huge \color{morado} Examen Julio 2016}
\author{\fontfamily{pag}\selectfont \bfseries \LARGE Marta Gómez y Braulio Vargas}

\begin{document}
\maketitle

\renewcommand{\tablename}{Tabla}

\begin{description}
    \item[Ejercicio 1] \textit{Factoriza, usando el \textbf{\textcolor{morado}{$\rho$ de Pollard}} el entero 299.}

    \begin{center}
    \begin{tabular}{c | c | c | c | c | c}
    $i$ & $a$ & $x$ & $y$ & $y-x$ & $mcd (y-x,n)$ \\
    \hline
    & $243$ & & & & \\
    $1$ & & $260$ & $147$ & $-113$ & $1$ \\
    $2$ & & $89$ & $80$ & $-9$ & $1$ \\
    $3$ & & $87$ & $292$ & $205$ & $1$ \\
    $4$ & & $77$ & $205$ & $128$ & $1$ \\
    $5$ & & $261$ & $183$ & $-78$ & \textbf{\textcolor{morado}{13}}
    \end{tabular}
    \end{center}

    \begin{displaymath}
        \frac{299}{13} = \mathbf{\textcolor{morado}{23}} \qquad\ 299 = 13 \cdot 23
    \end{displaymath}

    \item[Ejercicio 2] \textit{Encuentra un \textbf{\textcolor{morado}{cuerpo finito con 16 elementos}}. Calcula $1010 \times 0100$.}

    Para construir un cuerpo con $16 = 2^4$ elementos, necesitamos constuir $\mathbb{F}_{16}$ y para ello, necesitamos un polinomio de grado $n = 4$ que sea primitivo: $x^4 + x + 1$. Por tanto, vamos a trabajar en el cuerpo $\mathbb{Z}_{2_{x^4 + x + 1}}$. Podemos representar cada elemento del conjunto como una cadena de 4 bits.

    Para realizar la suma, basta con hacer la operación $\oplus$ bit a bit. En cambio, para la multiplicación, hay que realizar el producto en notación polinomial y posteriormente dividir por $x^4 + x + 1$.

    Para multiplicar $x^3 + x$ y $x^2$ operamos como sigue:

    \begin{displaymath}
        \frac{
            \begin{array}[b]{r}
            1\; 0\; 1\; 0 \\
            \times 0\; 1\; 0\; 0
            \end{array}
        }{\frac{
            \begin{array}[b]{r}
                0\; 0\; 0\; 0 \;\;\;\;\;\;\;\\
                0\; 0\; 0\; 0\; \;\;\;\;\;\;\;\;\;\\
                1\; 0\; 1\; 0\; \;\;\;\;\;\;\;\;\;\;\;\;\\
                0\; 0\; 0\; 0\; \;\;\;\;\;\;\;\;\;\;\;\;\;\;\;
            \end{array}
        }{
        \begin{array}[b]{r}
        0\; 1\; 0\; 1\; 0\; 0\; 0\; \;\;\;\;\;\;\;
        \end{array}
        }}
    \end{displaymath}

    Por tanto, $x^3 + x \times x^2 = x^5 + x^3$, pero al ser el grado del polinomio mayor a $n$, debemos reducirlo diviendo por $x^4 + x + 1$.

    \begin{displaymath}
        \frac{
            \begin{array}[b]{r}
                1\; 0\; 1\; 0\; 0\; 0\;\\
                1\; 0\; 0\; 1\; 1\; 0\;
            \end{array}
        }{
            \begin{array}[b]{r}
                0\; 0\; 1\; 1\; 1\; 0
            \end{array}
        }
    \end{displaymath}

    Por tanto, la solución final es \textcolor{morado}{$\mathbf{x^3 + x^2 + x}$}.

    \item[Ejercicio 3] \textit{Sea $\alpha$ un \textbf{\textcolor{morado}{elemento primitivo}} de $\mathbb{Z}_{59}^*$. Encuentra \textbf{\textcolor{morado}{$\log_{\alpha} 31$ en $\mathbb{Z}_{59}$}}.}

    En primer lugar, buscamos un elemento primitivo.

    \begin{center}
    \begin{tabular}{c | c | c}
    $i$ & $i^{\frac{n-1}{2}} \pmod n$ \\
    \hline
    $2$ & $58$  
    \end{tabular}
    \end{center}

    Por tanto, $\alpha = 2$.

    Ahora pasamos a calcular $\log_2 31 \pmod{59}$. Como $7 < \sqrt{59} < 8$, establecemos $s = 8$ y pasamos a calcular el paso enano.

    \begin{center}
    \begin{tabular}{c | c | c | c | c | c | c | c}
    0 & 1 & 2 & 3 & 4 & 5 & 6 & 7 \\
    \hline 
    31 & 3 & 6 & 12 & 24 & 48 & 37 & 15
    \end{tabular}
    \end{center}

    Y ahora pasamos a calcular el paso gigante, hasta encontar un elemento que coincida:

    \begin{center}
    \begin{tabular}{c | c | c | c | c | c | c | c}
    1 & 2 & 3 & 4 & 5 & 6 & 7 \\
    \hline 
    20 & 46 & 35 & 51 & 17 & 45 & 15 
    \end{tabular}
    \end{center}

    Paramos en $i=7$ porque ya hemos encontrado un elemento que se encuentre en el paso enano, $15$. Para calcular el logaritmo hacemos:

    \begin{displaymath}
        i \cdot s - posicion(paso\_enano, 15) = \mathbf{\textcolor{morado}{49}}
    \end{displaymath}

    \item[Ejercicio 4] \textit{Encuentra las \textcolor{morado}{\textbf{raíces cuadradas}} de $5$ en $\mathbb{Z}_{209}$. Haz lo mismo con $55$.}

    En primer lugar, factorizamos $n = 209$ en $p = 11$ y $q = 19$.

    Empezamos comprobando si $5$ tiene raíces en $p$ y en $q$.

    \begin{displaymath}
        \left(\frac{5}{19} \right) = (-1)^\frac{18 \cdot 4}{4} = 1 \qquad\ \left(\frac{5}{11} \right) = (-1)^\frac{10 \cdot 4}{4} = 1
    \end{displaymath}

    5 tiene raíces cuadradas tanto módulo $p$ como $q$. Necesitamos expresar $p-1$ y $q-1$ como $2^u \cdot s$  y encontrar un $n$ tal que $\left(\frac{n}{p} \right) = -1$ y $\left(\frac{n}{q} \right) = -1$.

    \begin{displaymath}
        18 = 2^1 \cdot 9 \; (u=1, \; s=9) \qquad\ 10 = 2^1 \cdot 5 \; (u=1, \; s=5)
    \end{displaymath}

    \begin{displaymath}
        \left(\frac{2}{19}\right) = -1 \qquad \left(\frac{2}{11}\right) = -1
    \end{displaymath}

    En ambos casos, $n = 2$. Como hemos obtenido $u = 1$ en ambos casos, una de las raíces será $r = a^{\frac{p+1}{4}}$ y la otra raíz será restar $p - r$ y $q - r$.

    \begin{displaymath}
        r = 5^5 \pmod p = 9 \qquad\ r = 5^3 \pmod q = 4
    \end{displaymath}
    \begin{displaymath}
        r' = 19 - 9 = 10 \qquad r' = 11 - 4 = 7
    \end{displaymath}

    Por tanto, ya tenemos las raíces cuadradas de 5, que son:

    \begin{displaymath}
    \begin{matrix}
        \sqrt{5} \in \mathbb{Z}_p & = & \{9,10\} \\
        \sqrt{5} \in \mathbb{Z}_q & = & \{4,7\}
    \end{matrix}
    \end{displaymath}

    Ahora pasamos a resolver el siguiente sistema de congruencias que resolvemos usando el Teorema Chino de los Restos:

    \begin{displaymath}
    \Bigg\{ \begin{matrix}
    r_1 & \equiv & 4 \bmod 11 \\
    r_1 & \equiv & 9 \bmod 19 
    \end{matrix}
    \qquad\
    \Bigg\{ 
    \begin{matrix}
    r_3 & \equiv & 7 \bmod 11 \\
    r_3 & \equiv & 9 \bmod 19
    \end{matrix}
    \end{displaymath}

    \begin{displaymath}
        r_1 = 4 + 11 \cdot l \qquad\ r_3 = 7 + 11 \cdot l
    \end{displaymath}
    \begin{displaymath}
        4 + 11 \cdot l \equiv 9 \bmod 19 \qquad\ 7 + 11 \cdot l \equiv 9 \bmod 19
    \end{displaymath}

    \begin{minipage}{0.5\textwidth}
    \begin{displaymath}
    \begin{matrix}
    11 \cdot l & \equiv & (9 - 4) \bmod 19 \\
    l & \equiv & (9 - 4) \cdot 11^{-1} \bmod 19\\
    l & \equiv & 5 \cdot 7 \bmod 19\\
    l & \equiv & 16 \bmod 19\\
    l & = & 16 + 19 \cdot s \\
    \end{matrix}     
    \end{displaymath}
    \end{minipage}
    \begin{minipage}{0.5\textwidth}
    \begin{displaymath}
    \begin{matrix}
    11 \cdot l & \equiv & (9 - 7) \bmod 19 \\
    l & \equiv & (9 - 7) \cdot 11^{-1} \bmod 19\\
    l & \equiv & 2 \cdot 7 \bmod 19\\
    l & \equiv & 14 \bmod 19\\
    l & = & 14 + 19 \cdot s
    \end{matrix}     
    \end{displaymath}
    \end{minipage}

    A partir de aquí, las operaciones son en $\mathbb{Z}_{209}$: 

    \begin{displaymath}
        r_1 = 4 + 11 \cdot (16 + 19 \cdot s) \qquad\ r_3 = 7 + 11 \cdot (14 + 19 \cdot s)
    \end{displaymath}
    \begin{displaymath}
        r_1 = 4 + (16 \cdot 11 + 19  \cdot 11 \cdot s) \qquad\ r_3 = 7 + (14 \cdot 11 + 19 \cdot 11 \cdot s)
    \end{displaymath}
    \begin{displaymath}
        r_1 = 4 + (176 + 0s) \qquad\ r_3 = 7 + (154 + 0s)
    \end{displaymath}
    \begin{displaymath}
        r_1 = 4 + 176 \qquad\ r_3 = 7 + 154
    \end{displaymath}
    \begin{displaymath}
        r_1 = \mathbf{\textcolor{morado}{180}} \qquad\ r_3 = \mathbf{\textcolor{morado}{161}}
    \end{displaymath}


    Y las dos otras raíces serán los complementarios a estas calculadas.

    \begin{displaymath}
        r_2 = 209 - 180 =  \textcolor{morado}{\mathbf{29}} \qquad\ r_4 = 209 - 161 = \mathbf{\textcolor{morado}{48}}
    \end{displaymath}

    En el caso de 55, debemos repetir la misma operación. En primer lugar, comprobamos si tiene raíces.

    \begin{displaymath}
        \left(\frac{55}{19} \right) = \left(\frac{55 \bmod 19}{19} \right) = \left(\frac{17}{19} \right) = (-1)^\frac{18 \cdot 16}{4} = 1 \qquad\ \left(\frac{55}{11} \right) = \left(\frac{55 \bmod 11}{11} \right) = \left(\frac{0}{11} \right) = 0
    \end{displaymath}

    55 no tiene raíces cuadradas módulo 11 y, por tanto, tampoco las tiene módulo 209.

    \item[Ejercicio 5] \textit{Sea $f: \mathbb{Z}_{19} \times \mathbb{Z}_8 \rightarrow \mathbb{Z}_{19}$ definida como $f(x,y) = 2^{xy}$. Utiliza la \textbf{\textcolor{morado}{construcción de Merkle-Damgård}} para construir a partir de $f$ una función resumen. Calcula la función resumen del mensaje (ya separado en bloques) $[2,1,0,3]$. Toma como vector inicial $7$ (valor inicial para $x$).}

    \begin{center}
        \begin{tabular}{c | c | c}
        $x$ & $y$ & $2^{xy} \bmod 19$ \\
        \hline
        7 & 2 & 6 \\
        6 & 1 & 7 \\
        7 & 0 & 1 \\
        3 & 1 & \textcolor{morado}{\textbf{8}}
        \end{tabular}
    \end{center}

    \item[Ejercicio 6] \textit{Escoge dos primos de dos dígitos. Construye una \textbf{\textcolor{morado}{llave pública y privada para RSA}} con esos dos primos. Si alguien te manda el valor 5 como mensaje cifrado con tu llave pública, ¿cuál es el \textbf{\textcolor{morado}{mensaje original}}?}

    Tomamos dos primos aleatorios en el intervalo $[10,99]$, por ejemplo $p = 79$ y $q = 23$. Calculamos $n = p\cdot q = 1817$ y también calculamos $\phi (n) = (p-1) \cdot (q-1) = 1716$. Una vez tenemos estos datos calculados, pasamos a calcular $e = 5$, primo relativo de $\phi (n)$, es decir $mcd (e, \phi(n)) = 1$. Por último, calculamos $d = e^{-1} \bmod \phi(n) = 1373$.

    Una vez tenemos nuestra clave pública $(n,e)$ y nuestra clave privada $d$, pasamos a descifrar el mensaje $c = 5$:

    \begin{displaymath}
        m = c^d \pmod n = 5^{1373} \pmod{1817} = 839
    \end{displaymath}

    Se puede comprobar que este descifrado es correcto, volviéndolo a cifrar:

    \begin{displaymath}
        c = m^e \pmod n = 839^5 \pmod{1817} = 5
    \end{displaymath}

    \item[Ejercicio 7] \textit{Supongamos que Jesús y Pedro andan últimamente muy ocupados, y como la ETSIIT está muy lejos, dedicen echar a suertes por teléfono quién vigila el examen de Criptografía y Computación este año. \textbf{\textcolor{morado}{Quien gane un reto de tirar una moneda por teléfono, vigila}}.}

    \textit{Pedro escoge los números $p = 742983794289739861$ y $q = 376487263921$. Y le manda a Jesús su producto $n = pq$. A continuación Jesús elige $u = 3498739487397$ y le envía $v = u^2 \bmod n$ a Pedro.}

    \textit{Pedro calcula entonces las cuatro raíces de $v$ módulo $n$,}

    \begin{displaymath}
        [3498739487397, 61408798834197592530916453239, 
    \end{displaymath}
    \begin{displaymath}   
        218315137015589671259824401742, 279723935849787260292001367584]
    \end{displaymath}

    \textit{y se queda con las dos que son menores que $n/2$}

    \begin{displaymath}
        [3498739487397, 61408798834197592530916453239]
    \end{displaymath}

    \textit{Finalmente Pedro escoge entonces $r = 61408798834197592530916453239$ y se lo manda a Jesús. ¿\textcolor{morado}{\textbf{Quién vigila el examen}}?}

    El reto lo gana \textcolor{morado}{\textbf{Jesús}}, porque, aunque Pedro ha calculado correctamente las raíces de $v$, ha enviado un $r$ distinto del $u$ inicial que eligió Jesús.
\end{description}

\end{document}