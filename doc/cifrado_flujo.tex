\documentclass[10pt,spanish]{article}

\usepackage[spanish]{babel}
\usepackage[utf8]{inputenc}
\usepackage{amsmath, amsthm}
\usepackage{amsfonts, amssymb, latexsym}
\usepackage{enumerate}
\usepackage[usenames, dvipsnames]{color}
\usepackage{colortbl}
\usepackage{minted}
\usepackage[left=3cm, right=3cm]{geometry}
\usepackage{cancel}
\usepackage{graphicx}
\usepackage{subfigure}

\usepackage[bookmarks=true,
            bookmarksnumbered=false, % true means bookmarks in
                                     % left window are numbered
            bookmarksopen=false,     % true means only level 1
                                     % are displayed.
            colorlinks=true,
            linkcolor=webblue]{hyperref}
\definecolor{webgreen}{rgb}{0, 0.5, 0} % less intense green
\definecolor{webblue}{rgb}{0, 0, 0.5}  % less intense blue
\definecolor{webred}{rgb}{0.5, 0, 0}   % less intense red

\setlength{\parindent}{0pt}
\setlength{\parskip}{1ex plus 0.5ex minus 0.2ex}

%%%%% Para cambiar el tipo de letra en el título de la sección %%%%%%%%%%%
\usepackage{sectsty}
\sectionfont{\fontfamily{pag}\selectfont}
\subsectionfont{\fontfamily{pag}\selectfont}
\subsubsectionfont{\fontfamily{pag}\selectfont}

\definecolor{rojo}{rgb}{0.8, 0.0, 0.0}
\definecolor{azul}{rgb}{0.01, 0.75, 0.24}
\definecolor{temacuatro}{rgb}{0.0, 0.62, 0.38}
\definecolor{temacinco}{rgb}{0.93, 0.35, 0.0}
\definecolor{temaseis}{rgb}{0.6, 0.4, 0.8}
\definecolor{temasiete}{rgb}{0.0, 0.47, 0.75}

% \usepackage[default]{frcursive}
% \usepackage[T1]{fontenc}

%Definimos autor y título
\title{\fontfamily{pag}\selectfont \bfseries \Huge \color{azul} Secuencias Pseudo-Aleatorias}
\author{\fontfamily{pag}\selectfont \bfseries \LARGE Marta Gómez}

\begin{document}
\maketitle

\renewcommand{\tablename}{Tabla}

\section{\textcolor{azul}Ejercicio 1}
\textbf{Escribe una función que determine si una secuencia de bits cumple los \textit{\textcolor{azul}{postulados de Golomb}}.}

Los \textit{\textcolor{azul}{postulados de Golomb}} son:

\begin{enumerate}[1.]
\item En todo período, la diferencia entre el número de unos y el número de ceros debe ser a lo sumo uno.

\begin{minted}[frame=single, label={Primer postulado}]{haskell}
n_ones  = fromIntegral $ sum s
n_zeros = (length s) - n_ones
cond1   = abs (n_ones - n_zeros) <= 1
\end{minted}

Para obtener el número de unos, calculo la sumatoria de la secuencia de bits 

\begin{displaymath}
    n_{unos} = \sum_{i=1}^n b_i
\end{displaymath}

y, para obtener el número de ceros, resto al tamaño total de la secuencia el número de unos.

\begin{displaymath}
    n_{ceros} = n - n_{unos}
\end{displaymath}

\item En un periodo, el número de rachas de longitud 1 debe ser el doble al número de rachas de longitud 2, y este a su vez, el doble de rachas de longitud 3, etc.

\begin{minted}[frame=single, label={Segundo postulado}]{haskell}
b       = map (\x -> length x) (group s)
n       = takeWhile (> 0) $ map (\x -> length $ elemIndices x b) [1..]
n_c2    = dropWhile (\x -> fst x >= 2*snd x) (zip n (snd (splitAt 1 n))) 
cond2   = length n_c2 <= 1
\end{minted}

En este caso, uso la función \texttt{group} para agrupar los bits iguales que están seguidos. Por ejemplo, si nuestra secuencia de bits, $s$, es $s=000111101011001$, la función \texttt{group} nos devolvería $\{000\},\{1111\},\{0\},\{1\},\{0\},\{11\},\{00\},\{1\}$.


Después, guardo el tamaño de cada grupo y calculo el número de elementos iguales (con igual tamaño). Siguiendo con la secuencia anterior, tendríamos guardados los siguientes tamaños: $\{3,4,1,1,1,2,2,1\}$ y tendríamos 

\begin{enumerate}[---]
    \item 4 grupos de tamaño 1
    \item 2 grupos de tamaño 2
    \item 1 grupo de tamaño 3
    \item 1 grupo de tamaño 4
\end{enumerate}

Por último, compruebo que el número de elementos iguales para cada tamaño de racha es, al menos, el doble que el tamaño del siguiente. Para darle algo de flexibilidad a esta condición, dejo que haya como máximo un par de elementos que no han cumplido la condición, es decir, que haya dos tamaños de racha con sólo un grupo.

\item La \textit{\textcolor{azul}{distancia de Hamming}} entre dos secuencias diferentes, obtenidas mediante desplazamientos circulares de un periodo, debe ser constante.

\begin{minted}[frame=single, label={Tercer postulado}]{haskell}
dists   = map (\x -> hamming_distance_one s x) [1..length s-1]
cond3   = (foldl1 (\ x y -> y - x) dists) == 0


rotar :: (Integral a) => [a] -> Int -> [a]
rotar s k = drop (length s - k) . take (2*(length s)-k) $ cycle s

hamming_distance_one :: (Integral a) => [a] -> Int -> a
hamming_distance_one s k = sum $ zipWith (\x y -> abs (x-y)) s (rotar s k)

\end{minted}

Con la función \texttt{rotar} hago un ciclo inverso. Por ejemplo, si la entrada de la función rotar es $s=0011101$ y $k=1$, la salida será $1001110$. Si $k=2$, la salida sería $0100111$.

La función \texttt{hamming\_distance\_one} calcula la \textit{\textcolor{azul}{distancia Hamming}} entre una secuencia de bits $s$ y esa misma secuencia rotada con un determinado $k$.

Para que la tercera condición se cumpla, la distancia de Hamming entre una secuecia y todas sus rotaciones debe ser igual. Por eso, hago un plegado restando distancias entre sí.
\end{enumerate}

La función \texttt{golomb} desarrollada es:

\begin{minted}[frame=single, label={Golomb}]{haskell}
golomb :: (Integral a) => [a] -> Bool
golomb s = cond1 && cond2 && cond3
    where
        n_ones  = fromIntegral $ sum s
        n_zeros = (length s) - n_ones
        cond1   = abs (n_ones - n_zeros) <= 1
        b       = map (\x -> length x) (group s)
        n       = takeWhile (> 0) $ map (\x -> length $ elemIndices x b) [1..]
        n_c2    = dropWhile (\x -> fst x >= 2*snd x) (zip n (snd (splitAt 1 n))) 
        cond2   = length n_c2 <= 1
        dists   = map (\x -> hamming_distance_one s x) [1..length s-1]
        cond3   = (foldl1 (\ x y -> y - x) dists) == 0
\end{minted}

\section{\textcolor{azul}Ejercicio 2}
\textbf{Implementa \textit{\textcolor{azul}{registros lineales de desplazamiento con retroalimentación}} (\textit{\textcolor{azul}{LFSR}}). La entrada son los coeficientes del polinomio de conexión, la semilla y la longitud de la secuencia de salida.}

\textbf{Ilustra con ejemplos la dependencia del periodo de la semilla en el caso de polinomios reducibles, la independencia en el caso de polinomios irreducibles y la maximalidad del periodo en el caso de polinomios primitivos.}

\textbf{Comprueba que los ejemplos con polinomios primitivos satisfacen los \textit{\textcolor{azul}{postulados de Golomb}}.}

\subsection{\textcolor{azul}Implementación}
El código desarrollado para implementar el LFSR es el siguiente:

\begin{minted}[frame=single, label={LFSR}]{haskell}
lfsr :: [Int] -> [Int] -> Int -> [Int]
lfsr c s n
    | length c /= length s = error $ "La semilla y los coeficientes del" ++
                                     " polinomio deben tener el mismo tamaño"
    | otherwise            = s ++ lst
        where
            seq = take (n - (length c - 1)) $ iterate (\x -> drop 1 (x ++ 
                [mod (sum $ zipWith (.&.) c x) 2])) s
            lst = map last $ tail seq
\end{minted}

Para calcular el nuevo bit que se añadirá a la secuencia, calculo un \textit{\textcolor{azul}{and}} bit a bit entre los coeficientes del polinomio y la secuencia. Una vez tengo la secuencia de bits resultado de la operación and, calculo su sumatoria módulo 2. 

Al tener \textit{\textcolor{azul}{Haskell}} la \textit{\textcolor{azul}{evaluación perezosa}}, sólo llamo a la función \texttt{iterate} $n$ veces (la longitud de la secuencia).

Vamos a ver un ejemplo detallado de cómo actuaría nuestra función. Supongamos $c(D) = D^4 + D^2 + 1$ y $s_0 = 1 \; s_1 = 0 \; s_2 = 0 \; s_3 = 1$. La secuencia de bits que obtendríamos con la función \texttt{iterate} sería la siguiente:

\begin{displaymath}
[[1,0,0,1],[0,0,1,1],[0,1,1,1],[1,1,1,1],[1,1,1,0],[1,1,0,0],[1,0,0,1]]
\end{displaymath}

Debido a que obtenemos una lista cuyo primer elemento es $s$ y los consecutivos son las secuencias de bits al ir aplicando la función, sólo necesitamos llamar a \texttt{take} con $n - tam(s)-1$. 

De esta lista de listas resultantes, sólo nos interesa el último elemento. Esto se implementa en la variable \texttt{lst}, que itera por las listas obtenidas quedándose únicamente con el último elemento:

\begin{displaymath}
[1,1,1,0,0,1]
\end{displaymath}

Por último, esta secuencia de bits obtenida se añade a la semilla inicial, $s$.

\end{document}