\documentclass[10pt,spanish]{article}

\usepackage[spanish]{babel}
\usepackage[utf8]{inputenc}
\usepackage{amsmath, amsthm}
\usepackage{amsfonts, amssymb, latexsym}
\usepackage{enumerate}
\usepackage[usenames, dvipsnames]{color}
\usepackage{colortbl}
\usepackage{minted}
\usepackage[left=3cm, right=3cm]{geometry}
\usepackage{cancel}
\usepackage{graphicx}
\usepackage{subfigure}

\usepackage[bookmarks=true,
            bookmarksnumbered=false, % true means bookmarks in
                                     % left window are numbered
            bookmarksopen=false,     % true means only level 1
                                     % are displayed.
            colorlinks=true,
            linkcolor=webblue]{hyperref}
\definecolor{webgreen}{rgb}{0, 0.5, 0} % less intense green
\definecolor{webblue}{rgb}{0, 0, 0.5}  % less intense blue
\definecolor{webred}{rgb}{0.5, 0, 0}   % less intense red

\setlength{\parindent}{0pt}
\setlength{\parskip}{1ex plus 0.5ex minus 0.2ex}

%%%%% Para cambiar el tipo de letra en el título de la sección %%%%%%%%%%%
\usepackage{sectsty}
\sectionfont{\fontfamily{pag}\selectfont}
\subsectionfont{\fontfamily{pag}\selectfont}
\subsubsectionfont{\fontfamily{pag}\selectfont}

\definecolor{azul}{rgb}{0.0, 0.44, 1.0}

% \usepackage[default]{frcursive}
% \usepackage[T1]{fontenc}

%Definimos autor y título
\title{\fontfamily{pag}\selectfont \bfseries \Huge \color{azul} Funciones de un sólo sentido}
\author{\fontfamily{pag}\selectfont \bfseries \LARGE Marta Gómez}

\begin{document}
\maketitle
\tableofcontents

\renewcommand{\tablename}{Tabla}

\section{\textcolor{azul}Ejercicio 1}
\textit{Sea $(a_1,\cdots,a_k)$ una secuencia \textbf{\textcolor{azul}{super-creciente}} de números positivos (la suma de todos los términos que preceden a $a_i$ es menor que $a_i$ para todo $i$). Elige un $n > \sum a_i$ y $u$ un entero positivo tal que $gcd(n,u) = 1$. Define $a_i^* = ua_i \pmod n$. La función \textbf{\textcolor{azul}{mochila}} (\textbf{\textcolor{azul}{knapsack}}) asociada a $(a_1^*,\cdots,a_k^*)$ es}

\begin{displaymath}
    f: \mathbb{Z}_2^k \rightarrow \mathbb{N}, f(x_1,\cdots,x_k) = \sum_{i=1}^k x_i a_i^*
\end{displaymath}

\textit{Implementa esta función y su inversa. La llave púlica es $(a_1^*,\cdots,a_k^*)$, mientras que la privada (y la puerta de atrás es) $((a_1,\cdots,a_k),n,u)$.}

\subsection{\textcolor{azul}Generación de una secuencia super-creciente aleatoria}
Para generar una secuencia \textit{\textcolor{azul}{super-creciente}} aleatoria, debemos asegurarnos de que dicha secuencia cumpla con la siguiente condición:

\begin{displaymath}
    b_i > \sum_{j=1}^{i-1} b_i \qquad\ \forall i, 2 \leq i \leq n
\end{displaymath}

Mi implementación en Haskell toma un primer elemento de forma aleatoria en el intervalo $[2,20]$ y, construye el resto de elementos a partir del primero, multiplicando el último elemento de la lista por dos. Como parámetro, recibe el número de elementos que tendrá la secuencia.

\begin{minted}[frame=single, label={genera\_secuencia}]{haskell}
genera_secuencia :: (Integral a, Random a) => a -> [a]
genera_secuencia t = take (fromIntegral t) $ iterate (\x -> x*2) r
    where
        r = fst $ randomR (2,20) $ mkStdGen (238012)
\end{minted}

Un ejemplo de secuencia obtenida es la siguiente, que empieza por 12:

\begin{displaymath}
    [12,24,48,96,192,384,768,1536,3072,6144]
\end{displaymath}

\subsection{\textcolor{azul}Generación de claves}
Para generar claves me he basado en el capítulo 4.5 del libro \textit{\textcolor{azul}{Notes on Cryptography}}. El código Haskell desarrollado es:

\begin{minted}[frame=single, label={Generación de claves}]{haskell}
is_prime_relative :: (Integral a) => a -> a -> Bool
is_prime_relative a b = x == 1
    where
        (x,_,_) = extended_euclides a b

mochi_gen_claves :: (Integral a, Random a) => [a] -> ([a], a, a, [a])
mochi_gen_claves s = (a,n,u,s)
    where
        n  = (sum s) * 2
        u  = head $ dropWhile (\x -> not (is_prime_relative x n)) $ randomRs (1,n-1) 
             $ mkStdGen (28165137)
        a  = map (\x -> x*u `mod` n) s
\end{minted}

En primer lugar, calculo $n = 2 \cdot \sum s_i$. En lugar de calcular un número aleatorio mayor a la sumatoria, calculo el doble de ésta porque en Haskell los aleatorios son algo complicados. 

Una vez he calculado $n$, paso a calcular $u$ generando números aleatorios hasta obtener uno que sea primo relativo con $n$.

Por último, calculo $a^*$ de la siguiente forma:

\begin{displaymath}
    a_i^* = ua_i \pmod n \qquad\ \forall i=1,\ldots,k 
\end{displaymath}

Devuelvo una tupla de cuatro elementos: el primero es la clave pública, $a^*$, y los tres siguientes forman la clave privada, $n, u, s$.

\subsection{\textcolor{azul}Cifrado}
Para cifrar un mensaje necesitamos la clave pública, $a^*$, generada previamente y el mensaje a cifrar en texto plano. 

En primer lugar, dicho mensaje debe ser convertido a una secuencia de bits, para esto uso las funciones desarrolladas en la práctica anterior. Una vez hecho, dicha secuencia de bits debe dividirse en trozos de igual tamaño que la clave pública (en caso de que el tamaño de la secuencia y la llave pública no sean múltiplos, se tendrán que añadir 0s al final de la secuencia). Al finalizar este proceso, el mensaje estará listo para ser cifrado.

Pasamos a resolver el problema de la mochila, seleccionando los objetos cuyo peso sea un bit con valor 1 en la secuencia de bits:

\begin{displaymath}
    b^* = \sum e_i \cdot a_i^*
\end{displaymath}

Esto nos dará una lista con la solución al problema de la mochila obtenida para cada grupo de bits. Esta lista es nuestro mensaje cifrado.

El código desarrollado en Haskell es:

\begin{minted}[frame=single,label={Cifrado de la mochila}]{haskell}
mochi_cifrado :: (Integral a, Random a) => [a] -> String -> [a]
mochi_cifrado s msg = f
    where
        b = binary_encoding msg
        t = length s
        z = (t - (length b `mod` t)) `mod` t
        c = b ++ replicate z 0
        d = chunksOf t c
        f = map (\x -> sum $ zipWith (\y z -> (fromIntegral y)*z) x s) d
\end{minted}

Por ejemplo, dado el mensaje ``hola'' y las siguientes claves:

\begin{enumerate}[\color{azul}{$\longrightarrow$}]
    \item $s = \{1,3,7,15,31,63,127,255\}$
    \item $n = 557$
    \item $u = 323$
    \item $a^* = \{323,412,33,544,297,360,486\}$
\end{enumerate}

El algoritmo seguiría los siguientes pasos:

\begin{enumerate}
    \item En primer lugar, transformar el string ``hola'' a binario:

\begin{minted}{haskell}
*Main> msg = "hola!"
*Main> b = binary_encoding msg
*Main> b
[0,1,1,0,1,0,0,0,0,1,1,0,1,1,1,1,0,1,1,0,1,1,0,0,0,1,1,0,0,0,0,1,0,0,1,0,0,0,0,1]
\end{minted}

    \item Como la longitud de $a^*$,8, es múltiplo de la longitud de la secuencia binaria, 32, no necesitamos añadir ningún 0 por detrás y directamente, dividimos la secuencia en subsecuencias de tamaño 8:

\begin{minted}{haskell}
*Main> t = length a
*Main> z = (t - (length b `mod` t)) `mod` t
*Main> z
0
*Main> d = chunksOf t b
*Main> d
[[0,1,1,0,1,0,0,0],[0,1,1,0,1,1,1,1],[0,1,1,0,1,1,0,0],[0,1,1,0,0,0,0,1],[0,0,1,0,0,0,0,1]]
\end{minted}

    \item Por último, calculamos el problema de la mochila usando los objetos que indiquen los bits a uno:

\begin{minted}{haskell}
*Main> f = map (\x -> sum $ zipWith (*) x a) d
*Main> f
[989,2132,1286,931,519]
*Main> 
\end{minted}
\end{enumerate}

\subsection{\textcolor{azul}Descifrado}
Para descifrar un mensaje, necesitamos la secuencia super-creciente original, $s$, además de $n$ y $u$. Los pasos para descifrar un mensaje son:

\begin{enumerate}
    \item Calcular la inversa $v$ de $u \pmod n$.

    \item Calcular $e^*$, para cada elemento del mensaje cifrado $b^*$:

    \begin{displaymath}
        e^*_i = v \cdot b^*_i \qquad\ \forall\; i=1,\ldots,k
    \end{displaymath}

    \item Resolver el problema de la mochila greedy (el algoritmo de \textit{\textcolor{azul}{Merkle Hellman}}) para cada $e^*_i$. Dicho algoritmo consiste en tomar el elemento de la lista mayor al peso restante por asignar.

    \item Pasar la secuencia binaria obtenida a texto plano.
\end{enumerate}

El código desarrollado en Haskell es:

\begin{minted}[frame=single,label={Descifrado de la mochila}]{haskell}
merkle_hellman :: (Integral a, Random a) => a -> [a] -> [Int]
merkle_hellman a [] = []
merkle_hellman a (xs:s)
    | a >= xs   = 1:merkle_hellman (a-xs) s
    | otherwise = 0:merkle_hellman a s 


mochi_descifrado :: (Integral a, Random a) => [a] -> a -> a -> [a] -> String
mochi_descifrado msg n u a = binary_decoding $ concat d
    where
        v   = inverse u n
        c   = map (\x -> x*v `mod` n) msg
        r   = reverse a
        d   = map (\x -> reverse $ merkle_hellman x r) c
\end{minted}

Siguiendo con el ejemplo anterior, una vez tenemos el mensaje $f = \{989,2132,1286,931,519\}$, los pasos que seguiría el algoritmo son:

\begin{enumerate}[1.]
    \item Calcular $v$:

\begin{minted}{haskell}
*Main> v = inverse u n
*Main> v
169
\end{minted}

    \item Calcular $e^*$:

\begin{minted}{haskell}
*Main> e = map (\x -> x*v `mod` n) f
*Main> e
[41,486,104,265,262]
\end{minted}

    \item Resolver el problema de la mochila usando el algoritmo Greedy:

\begin{minted}{haskell}
*Main> d = map (\x -> reverse $ merkle_hellman x (reverse s)) e
*Main> d
[[0,1,1,0,1,0,0,0],[0,1,1,0,1,1,1,1],[0,1,1,0,1,1,0,0],[0,1,1,0,0,0,0,1],[0,0,1,0,0,0,0,1]]
\end{minted}

    \item Transformar la secuencia de bits obtenida a caracteres:

\begin{minted}{haskell}
*Main> binary_decoding $ concat d
"hola!"
\end{minted}
\end{enumerate}

\section{\textcolor{azul}Ejercicio 2}

\textit{Sea $p$ un (pseudo-)primo mayor o igual que vuestro \textbf{\textcolor{azul}{número de identidad}}. Encuentra un \textbf{\textcolor{azul}{elemento primitivo}}, $\alpha$, de $\mathbb{Z}_p^*$; para facilitar el criterio, es bueno escoger $p$ de forma que $\frac{p-1}{2}$ sea también primo, y para ello usaremos Miller-Rabin. Definimos}

\begin{displaymath}
f: \mathbb{Z}_p \rightarrow \mathbb{Z}_p, \qquad\ x \rightarrow \alpha^x
\end{displaymath}

\textit{Calcula el \textcolor{azul}{\textbf{inverso de tu fecha de nacimiento}} con el formato AAAAMMDD.}

\subsection{\textcolor{azul}Calculando elementos primitivos}
Para encontrar una raíz primitiva de $p$, con $p$ y $\frac{p-1}{2}$ primos, me he basado en el criterio de los apuntes de Jesús García Miranda. Dados $p$ y $\alpha$, $\alpha$ será una raíz primitiva de $p$ si y sólo si $\alpha \pmod p \neq p-1$ y $\alpha^{\frac{p-1}{2}} \pmod p = p-1$. Debido a que realizo la búsqueda en el intervalo $[2,p-2]$, sólo compruebo la condición $\alpha^{\frac{p-2}{2}} \neq 1$.

\begin{minted}[frame=single, label={Encontrar elementos primitivos}]{haskell}
is_primitive_root :: (Integral a, Random a) => a -> a -> Bool
is_primitive_root p a = exponential_zn a p2 p == (p-1)
    where
        p2 = (p - 1) `div` 2

find_primitive_root :: (Integral a, Random a) => a -> a
find_primitive_root p = head $ dropWhile (\x -> not $ is_primitive_root p x) primos
    where
        primos = filter (miller_rabin) [2..p-2]
\end{minted}

\subsection{\textcolor{azul}Calculando el inverso de $f$}
Una vez hemos calculado $\alpha$, ahora pasamos a calcular el inverso de la función $f$ definida:

\begin{displaymath}
f: \mathbb{Z}_p \rightarrow \mathbb{Z}_p, \qquad\ x \rightarrow \alpha^x
\end{displaymath}

donde $x$ es una fecha de nacimiento en formato $AAAAMMDD$. El inverso de $f$ se corresponde con el \textit{\textcolor{azul}{Baby Step Giant Step}} que ya fue implementado en la práctica 1.

\begin{minted}[frame=single, label={Calculando el inverso de una fecha de nacimiento}]{haskell}
find_next_prime :: (Integral a, Random a) => a -> a
find_next_prime p = head $ dropWhile (\x -> not (miller_rabin x)) [p..p*2]

inverso_nacimiento :: (Integral a, Random a) => a -> a -> a
inverso_nacimiento id birthday = baby_s_giant_s a birthday p
    where
        p = find_next_prime p
        a = find_primitive_root p
\end{minted}

En mi caso, mi DNI es 75929776 y mi fecha de nacimiento, 19950207. El inverso de mi fecha de nacimiento es 67186975. Para comprobar que está bien, elevo $\alpha$ a dicho inverso, para obtener mi fecha de nacimiento.

\begin{minted}[frame=single, label={Inverso de mi fecha de nacimiento}]{haskell}
*Main> let id = 75929776
*Main> let birthday = 19950207
*Main> let i = inverso_nacimiento id birthday
*Main> i
67186975
*Main> let p = head $ dropWhile (\x -> not (miller_rabin x)) [id..id*2]
*Main> let a = find_primitive_root p
*Main> exponential_zn a i p
19950207
\end{minted}

\section{\textcolor{azul}Ejercicio 3}
\textit{En lo que sigue, $p$ y $q$ son enteros primos, y $n = p \cdot q$.}

\textit{Sea $f: \mathbb{Z}_n \rightarrow \mathbb{Z}_n$ la \textbf{\textcolor{azul}{función de Rabin}}: $f(x) = x^2$. Sea $n = 48478872564493742276963$. Sabemos que $f(12) = 144= f(37659670402359614687722)$. Usando esta información, \textbf{\textcolor{azul}{calcula $p$ y $q$}} (mirar la demostración del lema 2.43 del libro ``Lecture Notes on Cryptography'').}

Si $gcd(x-y,n)$ es un divisor no trivial de $n$ (por ejemplo, $p$), $q$ será simplemente dividir $\frac{n}{p}$. Como en el enunciado ya nos dan $x$ e $y$, ya lo tenemos todo hecho:

\begin{minted}[frame=single,label={Función para calcular p y q}]{haskell}
get_pq :: (Integral a, Random a) => a -> a -> a -> (a,a)
get_pq n x y = (p,q)
    where
        (p,_,_) = extended_euclides (x-y) n
        q       = n `div` p
\end{minted}

Con los datos del enunciado:

\begin{minted}[frame=single, label={Calculando p y q}]{haskell}
*Main> get_pq 48478872564493742276963 12 37659670402359614687722
(159497098847,303948303229)
*Main> (p,q) = get_pq 48478872564493742276963 12 37659670402359614687722
*Main> p*q == 48478872564493742276963
True
\end{minted}

\section{\textcolor{azul}Ejercicio 4}
\textit{Elige $a_0$ y $a_1$ dos cuadrados arbitrarios módulo $n$ ($n$ como en el Ejercicio 3). Sea}

\begin{displaymath}
    h: \mathbb{Z}_2 \times (\mathbb{Z}_n)^* \rightarrow (\mathbb{Z}_n)^*, \qquad\ h(b,x) = x^2 \cdot a_0^b \cdot a_1^{1-b}
\end{displaymath}

\textit{Usa la \textbf{\textcolor{azul}{construcción de Merkle-Damgård}} para implementar una función resumen tomando $h$ como función de compresión (esta $h$ fue definida por Goldwasser, Micali y Rivest). Los parámetros $a_0$, $a_1$ y $n$ se hacen públicos (la función debería admitir un parámetro en el que venga especificado el vector inicial).}

En primer lugar, he implementado una función que obtiene dos elementos y calcula su cuadrado módulo $n$. Dicha función también codifica un string a bits.

\begin{minted}[frame=single, label={gen\_md\_params}]{haskell}
gen_md_params :: (Integral a, Random a) => a -> String -> (a,a,[Int])
gen_md_params n s = (a0, a1, b)
    where
        b  = binary_encoding s
        l  = take 2 $ randomRs (1,n-1) $ mkStdGen (181876888)
        a0 = exponential_zn (head l) 2 n
        a1 = exponential_zn (last l) 2 n
\end{minted}

Una vez he generado los elementos necesarios, llamo a la siguiente función, que implementa el hash:

\begin{minted}[frame=single, label={Merkle Damgård}]{haskell}
merkle_damgard :: (Integral a, Random a) => (a, a, a) -> a -> [Int] -> a
merkle_damgard _ x []             = x
merkle_damgard (n,a0,a1) x (bs:b) = merkle_damgard (n,a0,a1) x' b
    where
        b1 = 1 - bs
        x' = ((x^2) * (a0^bs) * (a1^b1)) `mod` n
\end{minted}

Hay varios puntos a destacar sobre esta función:

\begin{enumerate}[\color{azul}{$\longrightarrow$}]
    \item No hace falta dividir la secuencia de entrada, pues el hash se realiza bit a bit.
    \item La función empieza calculando $h$ para el primer bit y un elemento inicial, $x$. Para los sucesivos bits, se usa el resultado obtenido en el bit anterior.

    \begin{displaymath}
        h_n (b_n, \cdots h_1(b_1, h_0(b_0, x)))
    \end{displaymath}

    \item Cuando ya no quedan más bits, se devuelve el resultado del último bit.
\end{enumerate}

Por ejemplo, el hash obtenido para mensaje ``hola'' es:

\begin{minted}[frame=single, label={Probando el hash}]{haskell}
*Main> let n = 48478872564493742276963
*Main> let (a0,a1,b) = gen_md_params n "hola"
*Main> merkle_damgard (n,a0,a1) 58 b
25379218943840191433672
\end{minted}

\section{\textcolor{azul}Ejercicio 5}
\textit{Sea $p$ el menor primo entero mayor o igual que tu número de identidad y sea $q$ el primer primo mayor o igual que tu fecha de nacimiento (AAAAMMDD). Selecciona $e$ tal que $gcd(e, (p-1)(q-1)) = 1$. Define la \textcolor{azul}{\textbf{función RSA}}}

\begin{displaymath}
    f: \mathbb{Z}_n \rightarrow \mathbb{Z}_n, \qquad\ x \rightarrow x^e
\end{displaymath}

\textit{Calcula el \textbf{\textcolor{azul}{inverso}} de $1234567890$.
}

Un mensaje en RSA se cifra usando el número $e$ calculado:

\begin{displaymath}
    c = m^e \pmod n \qquad\ n = p\cdot q
\end{displaymath}

El proceso de cifrado es \textit{\textcolor{azul}{inverso}} al de descifrado,

\begin{displaymath}
    m = c^d \pmod n
\end{displaymath}

Por tanto, cuando el enunciado pide calcular el inverso de un número, está pidiendo que lo ``descifremos''. Para ello, tenemos que calcular $d$:

\begin{displaymath}
    d \cdot e \equiv 1 \pmod{\phi (n)} \; \longrightarrow \; d = e^{-1} \pmod{\phi (n)}
\end{displaymath}

Mi implementación en Haskell:

\begin{minted}[frame=single, label={Inverso RSA}]{haskell}
inverso_rsa :: (Integral a, Random a) => a -> a -> a -> (a,a)
inverso_rsa id birthday msg = (c, exponential_zn c e n)
    where
        p     = find_next_prime id
        q     = find_next_prime birthday
        phi_n = (p-1)*(q-1)
        e     = head $ dropWhile (\x -> not $ is_prime_relative x phi_n) [2..phi_n]
        d     = inverse e phi_n
        n     = p*q
        c     = exponential_zn msg d n
\end{minted}

Calculo tanto el inverso de $1234567890$ como su ``cifrado'' para comprobar que la operación se realiza correctamente.

\begin{minted}[frame=single, label={Probando el inverso RSA}]{haskell}
*Main> inverso_rsa 20079894 19951128 1234567890
(133428684506750,1234567890)
*Main> 
\end{minted}

Como se ve, el inverso se ha calculado correctamente, pues devuelvo el mismo mensaje, $1234567890$.

\section{\textcolor{azul}Ejercicio 6}
\textit{Sea $n=50000000385000000551$, y sabemos que una inversa de $\mathbb{Z}_n \rightarrow \mathbb{Z}_n, \; x \rightarrow x^5$ es $x \rightarrow x^{10000000074000000101}$ (esto es, conoces tanto la llave pública como la privada de la función RSA). \textbf{\textcolor{azul}{Encuentra $p$ y $q$}} usando el método explicado en ``Notes on Cryptography''. Compara este método con el algoritmo de \textbf{\textcolor{azul}{Miller-Rabin}} y el ejercicio 3.}

El algoritmo implementado es el siguiente:

\begin{minted}[frame=single, label={Find p and q}]{haskell}
find_pq_rsa :: (Integral a, Random a) => a -> a -> a -> (a,a)
find_pq_rsa n e d
    | g /= 1     = (g, n `div` g)
    | abs y == 1 = (0,0)
    | otherwise  = find_pq_rsa_aux n (exponential_zn y 2 n) y
        where
            b       = snd $ descomposicion_2us (d*e - 1) 0
            x       = fst $ randomR (0,n) $ mkStdGen (51518732)
            (g,_,_) = extended_euclides x n
            y       = exponential_zn x b n

find_pq_rsa_aux :: (Integral a, Random a) => a -> a -> a -> (a,a)
find_pq_rsa_aux n (-1) z = (0,0)
find_pq_rsa_aux n 1 z    = (p,q)
    where
        (p,_,_) = extended_euclides (z+1) n
        (q,_,_) = extended_euclides (z-1) n
find_pq_rsa_aux n y z    = find_pq_rsa_aux n (exponential_zn y 2 n) y 
\end{minted}

Si el algoritmo falla, devolverá la tupla $(0,0)$.

\begin{minted}[frame=single, label={Probando el encontrar p y q}]{haskell}
*Main> find_pq_rsa 589 13 7
(19,31)
*Main> find_pq_rsa 50000000385000000551 5 10000000074000000101
(10000000019,5000000029)
*Main> 19*31
589
*Main> 10000000019*5000000029
50000000385000000551
*Main> 
\end{minted}

Como se ve, el algoritmo funciona correctamente. 

Tal y como se explica en el libro \textit{\textcolor{azul}{Notes on Cryptography}}, la probabilidad de factorizar $N$ escogiendo un $x$ aleatorio es de $1/2$, por tanto, ejecutando el algoritmo varias veces podríamos encontrar una solución.

En el ejercicio 3, necesitábamos tener un algoritmo que, dados $n$ y $a$ donde $a$ es un cuadrado módulo $n$, calculase $y$ tal que $a \equiv y^2 \pmod n$. En cambio, con este algoritmo necesitamos $e$ y $d$ tales que $e \cdot d \equiv 1 \pmod n$, es decir, que ambos sean inversos módulo $n$. En ambos ejercicios hemos obtenido estos datos de antemano.

En el caso del ejercicio 3, una vez obtenidos $x$ y $y$, la operación para obtener $p$ y $q$ es bastante sencilla. En este último caso, la operación es sencilla únicamente si $gcd(x,n) = 1$, esto es algo bastante raro, por lo que podemos decir que \textit{\textcolor{azul}{en cuanto a sencillez, gana el ejercicio 3}}.

Ahora bien, en el ejercicio 3 hay que estar escogiendo $x$ aleatorios hasta que se dé la condición mientras que en este último, una vez escogido un $x$ aleatorio que se usa para calcular el primer $y$, el resto es simplemente elevar $y$ al cuadrado, con la garantía de que sólo se hace $a$ veces como mucho, donde $d \cdot e - 1 = 2^a \cdot b$.

\end{document}