\documentclass[10pt,spanish]{article}

\usepackage[spanish]{babel}
\usepackage[utf8]{inputenc}
\usepackage{amsmath, amsthm}
\usepackage{amsfonts, amssymb, latexsym}
\usepackage{enumerate}
\usepackage[usenames, dvipsnames]{color}
\usepackage{colortbl}
\usepackage{minted}
\usepackage[left=3cm, right=3cm]{geometry}

\usepackage[bookmarks=true,
            bookmarksnumbered=false, % true means bookmarks in
                                     % left window are numbered
            bookmarksopen=false,     % true means only level 1
                                     % are displayed.
            colorlinks=true,
            linkcolor=webblue]{hyperref}
\definecolor{webgreen}{rgb}{0, 0.5, 0} % less intense green
\definecolor{webblue}{rgb}{0, 0, 0.5}  % less intense blue
\definecolor{webred}{rgb}{0.5, 0, 0}   % less intense red

\setlength{\parindent}{0pt}
\setlength{\parskip}{1ex plus 0.5ex minus 0.2ex}

%%%%% Para cambiar el tipo de letra en el título de la sección %%%%%%%%%%%
\usepackage{sectsty}
\sectionfont{\fontfamily{pag}\selectfont}
\subsectionfont{\fontfamily{pag}\selectfont}
\subsubsectionfont{\fontfamily{pag}\selectfont}

\definecolor{rojo}{rgb}{0.8, 0.0, 0.0}
\definecolor{azul}{rgb}{0.2, 0.2, 0.6}
\definecolor{temacuatro}{rgb}{0.0, 0.62, 0.38}
\definecolor{temacinco}{rgb}{0.93, 0.35, 0.0}
\definecolor{temaseis}{rgb}{0.6, 0.4, 0.8}
\definecolor{temasiete}{rgb}{0.0, 0.47, 0.75}

% \usepackage[default]{frcursive}
% \usepackage[T1]{fontenc}

%Definimos autor y título
\title{\fontfamily{pag}\selectfont \bfseries \Huge \color{rojo} Aritmética Modular}
\author{\fontfamily{pag}\selectfont \bfseries \LARGE Marta Gómez}

\begin{document}
\maketitle

\renewcommand{\tablename}{Tabla}

\section{\textcolor{rojo}Ejercicio 1}
\textit{Implementa el \textbf{\textcolor{rojo}{Algoritmo extendido de Euclides}} para el cálculo del máximo común divisor: dados dos enteros $a$ y $b$, encuentra $u, v \in \mathbb{Z}$ tales que $au + bv$ sea el máximo común divisor de $a$ y $b$.}

Para resolver este ejercicio, he desarrollado dos funciones en \textit{\textcolor{rojo}{Haskell}}:

\begin{enumerate}[---]
    \item \texttt{extended\_euclides}: esta función sirve como interfaz para el usuario y comprueba el caso base, que $b=0$. En el caso de que no se de el caso base, se pasa a utilizar la siguiente función.

\begin{minted}[linenos, frame=lines, label={extended\_euclides}]{haskell}
extended_euclides :: Integral a => a -> a -> (a, a, a)
extended_euclides a 0 = (a, 1, 0)
extended_euclides a b = extended_euclides_tabla a b 1 0 0 1

\end{minted}

    \item \texttt{extended\_euclides\_tabla}: esta  función implementa la tabla para el cálculo del algoritmo extendido de Euclides. A diferencia de la anterior, necesita como entrada los valores de $(x2, x1, y2, y1)$ además de $a$ y $b$. 

\begin{minted}[linenos, frame=lines, label={extended\_euclides\_tabla}]{haskell}
extended_euclides_tabla :: Integral a => a -> a -> a -> a -> a -> a -> (a, a, a)
extended_euclides_tabla a 0 x2 _ y2 _ = (a, x2, y2)
extended_euclides_tabla a b x2 x1 y2 y1 = extended_euclides_tabla b r x1 x y1 y
            where
                q = a `div` b
                r = a - q*b
                x = x2 - q*x1
                y = y2 - q*y1

\end{minted}
\end{enumerate}

\section{\textcolor{rojo}Ejercicio 2}
\textit{Usando el cálculo del ejercicio anterior, escribe una función que calcule $a^{-1}\bmod b$ para cualesquiera $a$ y $b$ que sean primos relativos.}

Para resolver este ejercicio he calculado el $u$ tal que $au + bv$. Esto es posible debido a que $a$ y $b$ son primos relativos y, por tanto, $mcd(a,b) = 1$. Si no se cumple esta condición, el programa devuelve $-1$.

Una vez obtenido $u$, he calculado su valor módulo $b$.

\begin{minted}[linenos, frame=lines, label={inverse}]{haskell}
inverse :: Integral a => a -> a -> a
inverse a b
    | r == 1 = i `mod` b
    | otherwise = -1
        where
            (r,i,_) = extended_euclides a b
\end{minted}

\section{\textcolor{rojo}Ejercicio 3}
\textit{Escribe una función que calcule $a^b \bmod n$ para cualesquiera $a, b \in \mathbb{Z}$ que sean primos relativos. La implementación debería tener en cuenta la representación binaria de $b$.}

Para hacer este ejercicio, podríamos primero realizar la potencia $a^b$ y después, calcular el módulo $n$ del resultado. Ahora bien, para un $b$ muy grande esta aproximación es muy costosa.

Es por eso que debe considerarse la representación binaria de $b = \sum_{i=0}^t b_i 2^i$. Usando dicha represetación binaria, $a^b$ puede calcularse como:

\begin{displaymath}
a^b = \prod_{i=0}^{t} a^{b_i2^i} = \left(a^{2^0} \right)^{b_0} \left(a^{2^1} \right)^{b_1} \cdots \left(a^{2^t} \right)^{b_t}
\end{displaymath}

El algoritmo da tantos pasos como dígitos tenga la representación binaria de $b$ pero la complejidad de los cálculos está limitada por $n$ y no perderemos tiempo haciendo una potencia muy grande para después aplicarle un módulo.

He desarrollado dos funciones Haskell para este ejercicio:

\begin{enumerate}[---]
    \item \texttt{exponential\_zn}: recibe como parámetros tres valores $(a, b, n)$ y comprueba los casos base, $b=0$ y $b=1$.

\begin{minted}[linenos, frame=lines, label={exponential\_zn}]{haskell}
exponential_zn :: Integral a => a -> a -> a -> a
exponential_zn _ 0 _ = 1
exponential_zn a 1 _ = a
exponential_zn a k n = exponential_zn_aux a k n 1 ki
                where
                    ki = k `mod` 2
\end{minted}

    \item \texttt{exponential\_zn\_aux}: realiza el cálculo de la potencia usando la representación binaria de $b$. Es importante destacar que no se calcula la representación binaria de $b$ sino que, en cada paso, se calcula el bit $b_i$ correspondiente.

\begin{minted}[linenos, frame=lines, label={exponential\_zn\_aux}]{haskell}
exponential_zn_aux :: Integral a => a -> a -> a -> a -> a -> a
exponential_zn_aux a 1 n b _ = a*b `mod` n
exponential_zn_aux a k n b ki
    | ki == 0   = exponential_zn_aux a0 k0 n b ki0
    | otherwise = exponential_zn_aux a0 k0 n b0 ki0
                where
                    a0  = a*a `mod` n
                    k0  = k `div` 2
                    ki0 = k0 `mod` 2
                    b0  = a*b `mod` n
\end{minted}
\end{enumerate}

\section{\textcolor{rojo}Ejercicio 4}
\textit{Dado un entero $p$, escribe una función para determinar si $p$ es (probablemente) primo usando el método de \textbf{\textcolor{rojo}{Miller-Rabin}}.}

El método de \textit{\textcolor{rojo}{Miller-Rabin}} tiene los siguientes pasos:

\begin{enumerate}[1.]
    \item En primer lugar, debemos descomponer $p -1 = 2^u \cdot s$. Para ello, dividimos $p-1$ entre $2$ hasta obtener resto $1$. $u$ es el número de divisiones que hemos hecho hasta obtener el resto $1$ (sin contar esa última) y $s$ es el cociente de la última división con resto $0$ que hemos hecho, debe ser \textbf{\textcolor{rojo}{impar}}.

\begin{minted}[linenos, frame=lines, label={descomposicion\_2us}]{haskell}
descomposicion_2us :: Integral a => a -> a -> (a, a)
descomposicion_2us p t 
    | m == 0    = descomposicion_2us u s
    | otherwise = (t, p)
                where
                    u = p `div` 2
                    s = t + 1
                    m = p `mod` 2   
\end{minted}

    \item Tomamos un $a$ aleatorio tal que $2 \leq a \leq p-2$. En este paso, se rompe la \textit{\textcolor{rojo}{transparencia referencial}} de Haskell, ya que cada vez que se ejecuta la función se obtiene un $a$ distinto:

\begin{minted}[linenos, frame=lines, label={Obteniendo un a random}]{haskell}
a = unsafePerformIO (randomRIO (2, p-2))
\end{minted}
    
    \item Calculamos el valor a $a=a^s$. Si es igual a $1$ o a $p-1$, $p$ será probablemente primo y termina la ejecución de la función.

    \item Si no ha sido así, iteramos desde $i=1$ hasta $i=u-1$ haciendo en cada paso $a = a^2$. Si en algún paso $a=1$, debemos comprobar que en el paso anterior obtuvimos $a=p-1$. Si esto se cumple, $p$ será probablemente primo y si no se cumple, no lo será.
\end{enumerate}

Todos estos pasos, están reflejados en las siguientes funciones:

\begin{minted}[linenos, frame=lines, label={Test de Miller-Rabin}]{haskell}
miller_rabin_once :: (Integral a, Random a) => a -> Bool
miller_rabin_once p
    | even p && p > 2             = False
    | (odd p && p <= 5) || p == 2 = True
    | b == 1 || b == (p-1)        = True
    | otherwise                   = miller_rabin_ok p u b 0 0
            where
                (u,s) = descomposicion_2us (p-1) 0
                a     = unsafePerformIO (randomRIO (2, p-2))
                b     = exponential_zn a s p

miller_rabin_ok :: (Integral a, Random a) => a -> a -> a -> a -> a -> Bool
miller_rabin_ok p u a i b
    | a == 1 && b == (p-1) = True
    | a == 1 && b /= (p-1) = False
    | i >= u               = False
    | otherwise            = miller_rabin_ok p u c (i+1) a
        where
            c = exponential_zn a 2 p
\end{minted}

Ahora bien, este test probabililístico tiene probabilidad de fallar $p_{error}^1 = \frac{1}{4}$. Por tanto, debemos repetir el test varias veces para obtener una probabilidad menor. En nuestro caso, lo hacemos 10 veces, obteniendo una probabilidad de error $p_{error}^{10} = \frac{1}{4^{10}}$.

\begin{minted}[linenos, frame=lines, label={Repetición del test 10 veces}]{haskell}
miller_rabin :: (Integral a, Random a) => a -> Bool
miller_rabin p = foldl1 (&&) (map miller_rabin_once (take 10 (repeat p))) 
\end{minted}

\section{\textcolor{rojo}Ejercicio 5}
\textit{Implementa el algoritmo \textbf{\textcolor{rojo}{paso enano-paso gigante}} para el cálculo de logaritmos discretos en $\mathbb{Z}_p$.}

Este algoritmo se basa en el cálculo de dos tablas, $S$ y $T$, para encontrar coincidencias entre ellas:

\begin{displaymath}
S = \{c, ca, ca^2, \cdots, ca^n\} \qquad\ T = \{a^2, a^{2s}, \cdots, a^{ts}, \cdots, a^{ss}\}
\end{displaymath}

Para hacer estas listas, he usado la función \texttt{zipWith} de Haskell. 

En el caso de la lista $S$, la he generado con el siguiente código:

\begin{minted}[linenos, frame=lines, label={Calculando la tabla S}]{haskell}
l1   = replicate (fromIntegral s) a
pa   = zipWith (\x y -> exponential_zn x y p) l1 [0..(fromIntegral s)]
tabS = zipWith (\x y -> x * y `mod` p) pa (replicate (fromIntegral s) c)
\end{minted}

en primer lugar, creo una lista con las potencias de $a$: $pa = \{a^0, a^1, \cdots, a^s\}$ y por último, la multiplico (módulo $p$) con una lista de $s$ elementos $c$.

En el caso de la lista $T$, la he generado con el siguiente código:

\begin{minted}[linenos, frame=lines, label={Calculando la tabla T}]{haskell}
as   = exponential_zn a (fromIntegral s) p
asl  = (replicate (fromIntegral s) as)
tabT = zipWith (\x y -> exponential_zn x y p) asl [1..(fromIntegral s)]
\end{minted}

en primer lugar, calculo el valor de $as=a^s \bmod p$ y creo una lista repitiendo $s$ veces $as$. Por último, calculo la lista $T$ usando la función $a^b \bmod n$ que he desarrollado en el ejercicio anterior.

Una vez tengo calculadas ambas listas, paso a buscar coincidencias entre ellas calculando su intersección.

\begin{displaymath}
T \cap S = \begin{cases}
\emptyset & \text{No hay solución} \\
\text{en otro caso} & k
\end{cases}  
\end{displaymath}

donde $k$ es el conjunto de todos los elementos que coinciden en ambas tablas.

\begin{minted}[linenos, frame=lines, label={Calculando la intersección de T y S}]{haskell}
i = intersect tabS tabT
\end{minted}

Una vez tenemos los elementos que coinciden en ambas tablas, necesitamos encontrar los índices de cada elemento en cada tabla, ya que el $k$ que buscamos se puede expresar como $k = t \cdot s - r$.  Para obtener los índices, $(t,r)$, de cada tabla he desarrollado la siguiente función Haskell:

\begin{minted}[linenos, frame=lines, label={getIndexTwoLists}]{haskell}
getIndexTwoLists :: Integral a => a -> [a] -> [a] -> (a,a)
getIndexTwoLists i t r = (it+1, ir)
            where
                it = (fromIntegral (fromJust (elemIndex i t)))
                ir = (fromIntegral (fromJust (elemIndex i r)))
\end{minted}

Una vez tengo los índices de cada elemento en las dos tablas, calculo para cada elemento $k$ su valor y devuelvo una lista con todos los $k$ encontrados:

\begin{minted}[linenos, frame=lines, label={Computing k}]{haskell}
indx = map (\x -> getIndexTwoLists x tabT tabS) i
k    = map (\x -> (fst x)*s - (snd x)) indx
\end{minted}

Todos estos pasos están reflejados en la función \texttt{shanks} que he desarrollado:

\begin{minted}[linenos, frame=lines, label={shanks}]{haskell}
shanks :: (Integral a, Random a) => a -> a -> a -> [a]
shanks a c p 
    | not $ miller_rabin p = error "p debe ser primo"
    | otherwise            = k
        where
            s    = ceiling (sqrt (fromIntegral p-1))
            l1   = replicate (fromIntegral s) a
            pa   = zipWith (\x y -> exponential_zn x y p) l1 [0..(fromIntegral s)]
            tabS = zipWith (\x y -> x * y `mod` p) pa (replicate (fromIntegral s) c)
            as   = exponential_zn a (fromIntegral s) p
            asl  = (replicate (fromIntegral s) as)
            tabT = zipWith (\x y -> exponential_zn x y p) asl [1..(fromIntegral s)]
            i    = intersect tabS tabT
            indx = map (\x -> getIndexTwoLists x tabT tabS) i
            k    = map (\x -> (fst x)*s - (snd x)) indx
\end{minted}

\end{document}
